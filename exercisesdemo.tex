\documentclass[10pt]{article}
\usepackage[top = 2.5cm, bottom = 2.5cm, left = 2.5cm, right = 2.5cm]{geometry} 
\usepackage[singlecounter]{structure}
\usepackage{fancyhdr}

\pagestyle{fancy}
\fancyhf{}
\lhead{\footnotesize MATH 3406: Homework 9}
\rhead{\footnotesize Connor Haynes}
\cfoot{\footnotesize \thepage}

\begin{document}
% first page

\thispagestyle{empty}

\begin{tabular}{p{15.5cm}}
  {\large \bf MATH 3406} \\ Georgia Institute of Technology \\ Fall 2024 \\ Dr. Zeitlin \\ \hline \\
\end{tabular}

\vspace{3mm}

\begin{center}
  {\Large \bf Homework 9}
  \vspace{2mm}

  {\bf Connor Haynes}
\end{center}

\vspace{4mm}

\begin{boxProb}
  (6A.1) Prove or give a counterexample: If $v_1,\dots,v_m\in V$, then
  \[\sum_{j=1}^{m}\sum_{k=1}^{m}\langle v_j,v_k\rangle\geq0.\]
  \solproof{Note that by right and left additivity respectively we find that
    \[\sum_{j=1}^{m}\sum_{k=1}^{m}\langle v_j,v_k\rangle = \sum_{j=1}^{m}\langle v_j,\sum_{k]1}^{m} v_k\rangle=\langle \sum_{j=1}^{m}v_j,\sum_{k=1}^{m}v_k\rangle\]
  and because $\sum_{j=1}^{m}v_j=\sum_{k=1}^{m}v_k$ we have that
  \[\sum_{j=1}^{m}\sum_{k=1}^{m}\langle v_j,v_k\rangle=\langle \sum_{j=1}^{m}v_j,\sum_{k=1}^{m}v_k\rangle\geq0.\]}
\end{boxProb}
\begin{boxProb}
  (6A.3)
    \begin{enumerate}
      \item Show that the function taking an ordered pair $((x_1,x_2),(y_1,y_2))$ of elements in $\mathbb{R}^2$ to $|x_1y_1|+|x_2y_2|$ is not an inner product on $\mathbb{R}^2$.
      \sol{Note that the function given does not satisfy homogeneity in the first slot for negative numbers.}
      \item Show that the function taking an ordered pair $((x_1,x_2,x_3),(y_1,y_2,y_3))$ to $x_1y_1+x_3y_3$ is not an inner product on $\mathbb{R}^3$.
      \sol{Note that this function does not satisfy that $\langle v,v\rangle=0$ if and only if $v=0$, as $v=(0,1,0)$ satisfies the equality.}
    \end{enumerate}
\end{boxProb}
\begin{boxProb}
  (6A.6) Suppose $u,v\in V$. Prove that $\langle u,v\rangle =0$ if and only if $||u||\leq ||u+av||$ for all $a\in\mathbb{F}$.
    \solproof{First the forward direction. By the fact that norms are positive semidefinite we have that showing $||u||\leq||u+av||$ is equivalent to showing that $||u||^2\leq||u+av||^2$. We may utilize the left and right additivity (and homogeneity) of the inner product to see that
      \[\langle u+av,u+av\rangle=\langle u,u\rangle+\langle u,av\rangle + \langle av,u\rangle + \langle av,av\rangle=\langle u,u\rangle + a^2\langle v,v\rangle=||u||^2+a^2||v||^2.\]
      Therefore $||u||^2\leq ||u+av||^2$.

      Now the backward direction. We prove the contrapositive. Suppose $\langle u,v\rangle\neq0$. Then
      \[||u+av||^2=||v||^2+||u||^2+a\langle u,v\rangle + a\langle v,u\rangle>||u||^2.\]
    Therefore the result holds.}
\end{boxProb}
\begin{boxProb}
  (6A.17) Prove that
    \[\left(\sum_{k=1}^{n}a_kb_k\right)^2\leq\left(\sum_{k=1}^{n}ka_k^2\right)\left(\sum_{k=1}^{n}\frac{b_k^2}{k}\right)\]
    for all real numbers $a_1,\dots,a_n$ and $b_1,\dots,b_n$.
    \solproof{Begin by noting that
      \[\left(\sum_{k=1}^{n}ka_k^2\right)\left(\sum_{k=1}^{n}\frac{b_k^2}{k}\right)=\sum_{i=1}^{n}\sum_{k=1}^{n}\frac{i}{k}a_i^2b_k^2=\sum_{k=1}^{n}a_k^2b_k^2+\sum_{k<i}^{n}\frac{i}{k}a_i^2b_k^2+\sum_{k>i}^{n}\frac{i}{k}a_i^2b_k^2.\]
      Now let $u=(a_1,\dots,a_n)$ and $v=(b_1,\dots,b_n)$ so that
      \[||u||^2||v||^2=\left(\sum_{k=1}^{n}a_k^2\right)\left(\sum_{k=1}^{n}b_k^2\right)=\sum_{i=1}^{n}\sum_{k=1}^{n}a_i^2b_k^2=\sum_{i=1}^{n}a_i^2b_i^2+\sum_{k\neq i}^{n}a_i^2b_k^2\]\[\leq\sum_{k=1}^{n}a_k^2b_k^2+\sum_{k<i}^{n}\frac{i}{k}a_i^2b_k^2+\sum_{k>i}^{n}\frac{i}{k}a_i^2b_k^2.\]
      Therefore
    \[\left(\sum_{k=1}^{n}a_kb_k\right)^2=|\langle u,v\rangle|\leq||u||^2||v||^2\leq\left(\sum_{k=1}^{n}ka_k^2\right)\left(\sum_{k=1}^{n}\frac{b_k^2}{k}\right).\]}
\end{boxProb}
\begin{boxProb}
  (6A.19) Suppose $v_1,\dots,v_n$ is a basis of $V$ and $T\in\mathcal{L}(V)$. Prove that if $\lambda$ is an eigenvalue of $T$, then
    \[|\lambda|^2\leq\sum_{j=1}^{n}\sum_{k=1}^{n}|\mathcal{M}(T)_{j,k}|^2.\]
    \solproof{Suppose $v$ is the eigenvector of $T$ corresponding to $\lambda$. Then $|\lambda|^2||v||^2=||\lambda v||^2=||Tv||^2$. Let $a_j$ denote the $j$-th row of $\mathcal{M}(T)$. Then $||Tv||^2=\sum_{j=1}^{n}|\langle v,a_j\rangle|^2$ and therefore $||Tv||^2\leq\sum_{j=1}^{n}||v||^2||a_j||^2$. Succinctly,
      \[|\lambda|^2||v||^2\leq\sum_{j=1}^{n}||v||^2||a_j||^2\]
      and we may divide both sides by $||v||^2$ to see that
      \[|\lambda|^2\leq\sum_{j=1}^{n}||a_j||^2=\sum_{j=1}^{n}\sum_{k=1}^{n}|a_{jk}|^2\]
      where $a_{jk}$ is the $k$-th element of row $j$. Equivalently,
   \[|\lambda|^2\leq\sum_{j=1}^{n}\sum_{k=1}^{n}|\mathcal{M}(T)_{j,k}|^2.\]}
\end{boxProb}
\begin{boxProb}
  (6A.21) Suppose $u,v\in V$ are such that
    \[||u||=3,\quad||u+v||=4,\quad||u-v||=6.\]
    What number does $||v||$ equal?
    \sol{By the parallelogram equality,
    \[16+36=2(9+||v||^2),\quad17=||v||^2,\quad||v||=\sqrt{17}.\]}
\end{boxProb}
\begin{boxProb}
  (6A.22) Show that if $u,v\in V$, then
    \[||u+v||||u-v||\leq ||u||^2+||v||^2.\]
    \solproof{By the parallelogram equality,
      \[||u+v||^2=2(||v||^2+||u||^2)-||u-v||^2,\quad||u-v||^2=2(||v||^2+||u||^2)-||u+v||^2.\]
      Now by the triangle inequality we have the following
      \[-\left(||u||^2+||v||^2+2||u||||v||\right)\leq-||u+v||^2,-||u-v||^2.\]
      Now we can write that
      \[-||u+v||^2\leq||u||^2+||v||^2-2||u||||v||\leq||u||^2+||v||^2\]
      for both $||u+v||^2$ and $||u-v||^2$. Therefore
    \[||u+v||^2||u-v||^2\leq(||u||^2+||v||^2)^2,\quad||u+v||||u-v||\leq||u||^2+||v||^2.\]}
\end{boxProb}
\begin{boxProb}
  (6A.26) Suppose $V$ is a real inner product space. Prove that
    \[\langle u,v\rangle=\frac{||u+v||^2-||u-v||^2}{4}\]
    for all $u,v\in V$.
    \solproof{Note that
      \[||u+v||^2=\langle u+v,u+v\rangle=\langle u,u\rangle + \langle v,v\rangle + 2\langle u,v\rangle,\]
      and
      \[||u-v||^2=\langle u-v,u-v\rangle=\langle u,u\rangle+\langle v,v\rangle -2\langle u,v\rangle.\]
      Therefore
      \[\frac{||u+v||^2-||u-v||^2}{4}=\frac{4\langle u,v\rangle}{4}=\langle u,v\rangle\]}
\end{boxProb}
\begin{boxProb}
  (6A.27) Suppose $V$ is a complex inner product space. Prove that
    \[\langle u,v\rangle=\frac{||u+v||^2-||u-v||^2+||u+iv||^2i-||u-iv||^2i}{4}\]
    for all $u,v\in V$.
    \solproof{Note that if $\langle u,v\rangle=a+bi$, then $\langle v,u\rangle=a-bi$ and
      \[||u+v||^2=\langle u,u\rangle + \langle v,v\rangle+\langle u,v\rangle + \langle v,u\rangle=||u||^2+||v||^2+2a,\]
      \[||u-v||^2=\langle u,u\rangle+\langle v,v\rangle-\langle u,v\rangle-\langle v,u\rangle=||u||^2+||v||^2-2a,\]
      \[||u+iv||=\langle u,u\rangle-\langle v,v\rangle+i\langle u,v\rangle+i\langle v,u\rangle=||u||^2-||v||^2+2b,\]
      \[||u-iv||=\langle u,u\rangle+\langle v,v\rangle-i\langle u,v\rangle-i\langle v,u\rangle=||u||^2+||v||^2-2b.\]
      Therefore
    \[\frac{||u+v||^2-||u-v||^2+||u+iv||^2i-||u-iv||^2i}{4}=\frac{4a+4bi}{4}=a+bi=\langle u,v\rangle.\]}

\end{boxProb}
\begin{boxProb}
  (6A.30) Suppose $V$ is a real inner product space. For $u,v,w,x\in V$, define
    \[\langle u+iv,w+ix\rangle_C=\langle u,w\rangle+\langle v,x\rangle+(\langle v,w\rangle-\langle u,x\rangle)i.\]
    \begin{enumerate}
      \item Show that $\langle \cdot,\cdot\rangle_C$ makes $V_C$ into a complex inner product space.
      \solproof{First we show that this proposed inner product is positive semi-definite for $\langle a+bi,a+bi\rangle_C$:
        \[\langle a+bi,a+bi\rangle_C=\langle a,a\rangle+\langle b,b\rangle+\left(\langle b,a\rangle - \langle a,b\rangle\right)i=||a||^2+||b||^2.\]
        Now we must show that $\langle 0,0\rangle_C=0$, which follows from the above argument:
        \[\langle 0,0\rangle_C=||0||^2+||0||^2=0.\]
        For left additivity note that
        \[\langle (a+bi)+(c+di),x+yi\rangle_C=\langle a+c+(b+d)i,x+iy\rangle\]\[=\langle a+c,x\rangle+\langle b+d,x\rangle+(\langle b+d,x\rangle+\langle a+c,y\rangle)i\]\[=\langle a,x\rangle + \langle c,x\rangle + \langle b,y\rangle + \langle d,y\rangle + (\langle b,x\rangle+\langle a,y\rangle)i+(\langle d,x\rangle+\langle c,y\rangle)i\]\[=\langle a+bi,x+yi\rangle + \langle c+di,x+yi\rangle.\]
        To show that the product satisfies left homogeneity, we write
        \[\langle\lambda(a+bi),x+yi\rangle_C=\langle \lambda a,x\rangle + \langle \lambda b,y\rangle + (\langle \lambda b,x\rangle + \langle \lambda a,y\rangle)i=\lambda (\langle a+bi,x+yi\rangle_C).\]
        What remains is to show that the product obeys conjugate symmetry. To see this, note that it follows from the fact that if
        \[\langle a+bi,c+di\rangle_C=\langle a,c\rangle + \langle b,d\rangle + (\langle b,c\rangle - \langle a,d\rangle)i,\]
        then the imaginary component of $\langle c+di,a+bi\rangle_C$ takes the form
        \[(\langle a,d\rangle - \langle b,c\rangle)i=-(\langle b,c\rangle - \langle a,d\rangle)i\]
      and so the two are indeed conjugates.}
      \item Show that if $u,v\in V$, then
      \[\langle u,v\rangle_C=\langle u,v\rangle\quad\text{and}\quad||u+iv||^2_C=||u||^2+||v||^2.\]
      \solproof{Note that if $u,v\in V$ then $\langle 0,v\rangle - \langle u,0\rangle=0$ and so $\langle u,v\rangle_C=\langle u,v\rangle$. Further,
      \[||u+iv||^2_C=\langle u+iv,u+iv\rangle_C=\langle u,u\rangle + \langle v,v\rangle + (\langle v,u\rangle - \langle u,v\rangle)i=||u||^2+||v||^2.\]}
    \end{enumerate}
\end{boxProb}

\end{document}
